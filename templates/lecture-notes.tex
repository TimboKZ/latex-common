%-- summary: Template for lecture notes with a lot of useful snippets.

\documentclass{article}

% Title for the `compact-header` snippet
\newcommand\HeaderTitle{\textbf{CS 124~~~Lecture Notes} by Timur Kuzhagaliyev}

%-- Rammy start ----------------
%-- summary: Common math symbols and shorthand commands.

% Symbol for evaluating function at a value, e.g. \evalat_{x_0}
\newcommand\evalat{\biggr\rvert}

% Shorthand for various symbols
\newcommand\x{\times}
\newcommand\ra{\rightarrow}
\newcommand\la{\leftarrow}

% Norm and inner product
\usepackage{amsmath}
\newcommand{\inner}{\langle\cdot,\cdot\rangle}
\newcommand{\inr}[1]{\langle #1\rangle}
\newcommand\norm[1]{\left\lVert #1 \right\rVert}

% Tensor product
\newcommand{\tens}{%
  \mathbin{\mathop{\otimes}}%
}

% Number sets
\usepackage{amsfonts}
\newcommand\W{\mathbb{W}}
\newcommand\N{\mathbb{N}}
\newcommand\M{\mathbb{M}}
\newcommand\Z{\mathbb{Z}}
\newcommand\I{\mathbb{I}}
\newcommand\Q{\mathbb{Q}}
\newcommand\R{\mathbb{R}}
\newcommand\C{\mathbb{C}}
 %-- snippet: symbols
%-- summary: Compact header for a document.

% The title for the header is specified using the \HeaderTitle command.
% If this command was not defined, the default value below will be used.
\providecommand\HeaderTitle{Untitled Document}

\usepackage{fancyhdr}
\usepackage[
  top=2cm,
  bottom=2cm,
  left=2cm,
  right=2cm,
  headheight=17pt, % as per the warning by fancyhdr
  includehead,includefoot,
  heightrounded, % to avoid spurious underfull messages
]{geometry} 
\pagestyle{fancy}
\lhead{\HeaderTitle}
\rhead{Page \thepage}
 %-- snippet: compact-header 
%-- summary: `hyperref` package with blue links.

\usepackage[colorlinks = true,
            linkcolor = blue,
            urlcolor  = blue,
            citecolor = blue,
            anchorcolor = blue]{hyperref}
 %-- snippet: urls
%-- summary: Collection of TeX packages for figure management.

\usepackage{graphicx}
\usepackage{float}
\usepackage{wrapfig}

 %-- snippet: figures
%-- summary: Packages and style adjustments for source code.

% Use a different monospace font.
\usepackage[scaled=1.05]{inconsolata}

\usepackage{listings}
\usepackage{color}
\usepackage{xcolor}

% `listings` package settings
\definecolor{codebg}{rgb}{0.96,0.96,0.96}
\definecolor{mygreen}{rgb}{0,0.6,0}
\definecolor{mygray}{rgb}{0.5,0.5,0.5}
\definecolor{mymauve}{rgb}{0.58,0,0.82}
\lstset{columns=fixed, basicstyle=\ttfamily, basewidth=0.485em, %
  backgroundcolor=\color{codebg},   % choose the background color
  breaklines=true,                 % automatic line breaking only at whitespace
  captionpos=b,                    % sets the caption-position to bottom
  commentstyle=\color{mygreen},    % comment style
  keywordstyle=\color{blue},       % keyword style
  stringstyle=\color{mymauve},     % string literal style
  extendedchars=true,
  showstringspaces=false,
  showspaces=false,
  frame=single,
  framesep=8pt,
  xleftmargin=8pt,
  xrightmargin=8pt,
  framerule=0pt,
}

% Define JavaScript language highlighting
\lstdefinelanguage{javascript}{
  keywords={typeof, new, true, false, catch, function, return, null, catch, switch, var, if, in, while, do, else, case, break},
  keywordstyle=\color{blue}\bfseries,
  ndkeywords={class, export, boolean, throw, implements, import, this},
  ndkeywordstyle=\color{darkgray}\bfseries,
  identifierstyle=\color{black},
  sensitive=false,
  comment=[l]{//},
  morecomment=[s]{/*}{*/},
  morestring=[b]',
  morestring=[b]"
}

\setlength{\fboxsep}{2pt}
\newcommand{\code}[1]{\colorbox{codebg}{\ttfamily #1}}

 %-- snippet: code
%-- summary: Convenient miscellaneous commands.

\usepackage[utf8]{inputenc}
\usepackage[normalem]{ulem}
\usepackage{color}

% Easy way to specify TODOs in the text.
\newcommand{\TODO}[1]{\textit{\color{red} \textbf{TODO:} #1}}
\newcommand{\todo}[1]{\textit{\color{red} \textbf{TODO:} #1}}

% Aliases for bold and italic text
\providecommand\bold{}
\renewcommand{\bold}[1]{\textbf{#1}}
\providecommand\italic{}
\renewcommand{\italic}[1]{\textit{#1}}

% A command to mark the start of a new lecture or slide (useful for lecture
% notes).  Draws a line and prints the number of the slide.
\newcommand{\lecturemark}[2]{\vspace{20px}\noindent{\color{gray}\bold{Lecture
#1} #2 \sout{\hfill}}\vspace{10px}}
\newcommand{\slidemark}[1]{\vspace{10px}\noindent{\color{gray}\bold{Slide #1}
}\vspace{5px}}

% Use sans-serif fonts
\def\rmdefault{bch}

 %-- snippet: misc
%-- Rammy end ------------------

% Don't wrap long matrices
\setcounter{MaxMatrixCols}{20}

\begin{document}


%-- example usage start
\begin{center}
{\LARGE CS 124~~~Operating Systems}
\vspace{10px}\\
\end{center}

\section{Lecture notes template}
This template was meant to be used for lecture notes. It has a lot of useful
shorthand commands, packages and settings defined to make lecture note writing
an effortless process. Refer to section \ref{sec:features} for some examples.


\section{Features}\label{sec:features}
Most of the features of this template come from different TeX snippets defined
in the \code{latex-common} Rammy module. Check
\href{th://github.com/TimboKZ/latex-common}{its GitHub page} for the full list
of a available snippets.

\subsection{Snippet: \texttt{compact-header}}
The small header you see at the top of this page is a part of this template.
You can customise its content by (re)defining the \code{\textbackslash
HeaderTitle} command.

This snippet also defines the page margins.


\subsection{Snippet: \texttt{symbols}}

This template includes a lot of useful Math symbols. Here's a quick preview:

\[
\begin{array}{lll}%
    \R^n \x \R^m & \qquad \M_{r \x k} \qquad & \C, \Z, \Q \\
    \\
    f: \R^n \ra \R & \qquad \frac{df}{dt}\evalat_{t=4} \qquad & A \tens B \\
    \\
    \inr{v, u} & \qquad \inner \qquad & \norm{A} \\
\end{array}
\]

See the source code of the \code{symbols} snippet for more info.


\subsection{Snippet: \texttt{code}}

As the name implies, this snippet adds a bunch of code-related features.
Code snippets should be inserted using the \code{lstlisting}} environment:

\begin{lstlisting}[language=javascript]
/**
  * This code snippet was taken from Rammy CLI tests.
  */
describe('rammy init', () => {
    it('should succeed in an empty directory', () => {
        TU.removeFixture(configName);
        assertRammyCommand('init', 0);
        assert.isTrue(TU.hasFixture(configName), 'Config file was not created!');
    });
    it('should fail when a project has already been initialised', () => {
        assertRammyCommand('init', 1);
    });
});
\end{lstlisting}

Additionally, the \code{\textbackslash code} command can be used to insert
inline code snippets.


\subsection{Other snippets and features}

This template has coloured links:
\url{https://github.com/TimboKZ/latex-common}. It also supports long matrices:

\[
Z =
\begin{bmatrix}
    q_{sum1} & q_{sum2} & q_{sum3} & 0 & 0 & 0 & 0 & 0 & 0 & m & 0 & 0 \\
    0 & 0 & 0 & q_{sum1} & q_{sum2} & q_{sum3} & 0 & 0 & 0 & 0 & m & 0 \\
    0 & 0 & 0 & 0 & 0 & 0 & q_{sum1} & q_{sum2} & q_{sum3} & 0 & 0 & m \\
    x_{11} & x_{21} & x_{31} & 0 & 0 & 0 & 0 & 0 & 0 & q_{sum1} & 0 & 0 \\
    x_{12} & x_{22} & x_{32} & 0 & 0 & 0 & 0 & 0 & 0 & q_{sum2} & 0 & 0 \\
    \hdotsfor{12} \\
    \hdotsfor{12} \\
    \hdotsfor{12} \\
    \hdotsfor{12} \\
    \hdotsfor{12} \\
    \hdotsfor{12} \\
    0 & 0 & 0 & 0 & 0 & 0 & x_{13} & x_{23} & x_{33} & 0 & 0 & q_{sum3} \\
\end{bmatrix}
\]

This template doesn't use standard $\LaTeX$ fonts - it uses a sans-serif font for
the main text and Inconsolota as the monospace font. The packages needed for
common figure functionality are also included via the \code{figures} snippet.
%-- example usage end


\end{document}
