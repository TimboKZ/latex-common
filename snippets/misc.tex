%-- summary: Convenient miscellaneous commands.

\usepackage[utf8]{inputenc}
\usepackage[normalem]{ulem}
\usepackage{color}

% Easy way to specify TODOs in the text.
\newcommand{\TODO}[1]{\textit{\color{red} \textbf{TODO:} #1}}
\newcommand{\todo}[1]{\textit{\color{red} \textbf{TODO:} #1}}

% Aliases for bold and italic text
\providecommand\bold{}
\renewcommand{\bold}[1]{\textbf{#1}}
\providecommand\italic{}
\renewcommand{\italic}[1]{\textit{#1}}

% A command to mark the start of a new lecture or slide (useful for lecture
% notes).  Draws a line and prints the number of the slide.
\newcommand{\lecturemark}[2]{\vspace{20px}\noindent{\color{gray}\bold{Lecture
#1} #2 \sout{\hfill}}\vspace{10px}}
\newcommand{\slidemark}[1]{\vspace{10px}\noindent{\color{gray}\bold{Slide #1}
}\vspace{5px}}

% Use sans-serif fonts
\def\rmdefault{bch}

